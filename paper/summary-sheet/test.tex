% 文档类
\documentclass[13pt]{ctexart}
% 设置页面
\usepackage{geometry}
% 设置页眉页脚
\usepackage{fancyhdr}
% 清空页眉页脚
\pagestyle{fancy}
% 颜色
\usepackage{xcolor}
% 设置表格的列格式
\usepackage{array}
% 摘要页缩进
\usepackage{changepage}

\begin{document}
\newgeometry{top = 1cm, right = 2.54cm, left = 2.54cm, bottom = 2.54cm}
% 第一页的字体为times new roman
\setmainfont{Times New Roman}
\thispagestyle{empty}


\begin{table}[h]
    \quad { }  \begin{minipage}[t]{5.5cm}
        % arraystretch 是调节列高
        \begin{tabular}[t]{>{\centering\arraybackslash}b{10em}}
            \fontsize{12pt}{10pt}\selectfont \textbf{Problem Chosen}\\ [2pt]
            {\color{red} \fontsize{20pt}{10pt}\selectfont ABCDEF}
        \end{tabular}
    \end{minipage}
    \begin{minipage}[t]{5.2cm}
        \begin{tabular}[t]{>{\centering\arraybackslash}p{10em}}
            \fontsize{12pt}{10pt}\selectfont \textbf{2020} \\ [-2pt]
            \fontsize{12pt}{10pt}\selectfont \textbf{MCM/ICM} \\ [-2pt]
            \fontsize{12pt}{10pt}\selectfont \textbf{Summary Sheet}
        \end{tabular}
    \end{minipage}
    \begin{minipage}[t]{3cm}
        \begin{tabular}[t]{>{\centering\arraybackslash}b{12em}}
            \fontsize{12pt}{10pt}\selectfont \textbf{Team Control Number} \\ [2pt]
            {\color{red} \fontsize{21pt}{10pt}\selectfont 1111111}
        \end{tabular}
    \end{minipage}
\end{table}
\vspace{-20pt}
\noindent{\rule{\textwidth}{0.5mm}}

% 标题
{\centering\fontsize{18}{16}\selectfont\textbf{{Analysis and Supression of Opioid Spread}}
% 摘要
\vspace{10pt} 

\fontsize{13}{10}\selectfont\textbf{{Summary}}\par}

\vspace{10pt}

% 正文字体 13 pt
\fontsize{13}{12.5}\selectfont

\begin{adjustwidth}{1cm}{1cm}
\indent { }{ }{ }{ }{ }{ }We propose a model to describe the characteristics of the number, rate and direction of opioid spread in and between states and counties using transition matrices. The result shows that among the five given states, Ohio is most likely the source of opioid cases and has frequent opioid transition with Pennsylvania and Kentucky. We also find that if no effective regulatory measures are imposed, the number of drug cases in Ohio will increase up to 141,266 in the next decade. The spread of opioids between counties was analyzed in the same way, and the result is shown in Figure 3.

We determine two epidemic thresholds based on characteristics of opioid spread, with respect to the number of opioid cases and opioid spread rate respectively. We perform simulations to forecast the number of opioid cases in states and counties. The result shows that the number of narcotic analgesics cases in Ohio reaches the threshold in 2022, and the number of heroin cases in Pennsylvania reached the threshold in 2023.

We analyze the correlations between the number of opioid cases and various socio-economic factors using information entropy. We find that the number of opioids cases demonstrates high correlations with disability status, educational level, family status and adolescents. Therefore, we provide suggestions based on these factors to help suppress opioid spread. We also evaluate the effectiveness of these strategies on both the spread rate and the number of opioid cases. The result shows that when the amount of opioids spread is reduced to 74.5\%, the amount of opioids will stay on a relatively low level, and become lower than that of 2010 after 5 years.

We also do sensitivity analysis of various parameters to prove the robustness of our model. The result shows that both the continuous and periodical inflows lead to increase in the number of opioids in the state, and the increased amount depends on the specific transition amount.

\vspace{15pt}
\textbf{key words} : Transition matrix; Multi-level thresholds; Information entropy
\end{adjustwidth} 

\end{document}
